\documentclass[a4paper,12pt]{article}
\usepackage[italian]{babel}
\usepackage[utf8]{inputenc} %utf8 per i catatteri accentati
\usepackage{hyperref}% links interaction nel documento
\usepackage{cite}
%INSERISCI IMMAGINI
%\usepackage{graphicx}
%\graphicspath{ {./src/} } %path grafiche
\pagenumbering{Roman}
%il comando \virgo implementa la funzione "" 
\newcommand{\virgo}[1]{``#1''}
\topmargin 2pt
%se la 'date' non e' commentato, scrive data odierna
%\date {11 Febbraio 2017}

%\bibliographystyle{alphanum}
%\bibliography{biblio}

\title{Una percezione del silenzio}

\author{Bruno Curtolo \\
conservatorio \textit{A. Steffani} \\
\emph{(Castelfranco Veneto)}\\
\\
docente corso \textit{prof. Federico Pelle}}

\begin{document}

\maketitle

\newpage
\section*{Una versione percettiva}
I processi creativi e le intuizioni dei compositori plasmano quelle caratteristiche formali che contraddistinguono ogni opera d'arte.\\
Se la struttura tematica, rappresenta il riferimento classico, tradizionale, dell'\textit{idea} musicale\cite{rif1};
\\
\begin{small}\\
la scienza, gettando nuove luce sulla natura, permette alla musica di progredire[...] rivelando ai nostri sensi armonie e sensazioni mai provate prima. Sulla soglia del Bello, arte e scienza collaborano \textit{[E. Varese - 1983]\cite{rif2}}
\end{small}
\\
\\
Le frontiere sperimentali han dunque alimentato l'arte emancipando la ricerca a nuovi linguaggi e tecniche compositive: la \mbox{dodecafonia} ed il \mbox{serialismo},
\mbox{l'indeterminatezza} e
le leggi della \mbox{probabilità},
l'utilizzo di strutture subliminali che regolarizzano i parametri musicali in categorie \mbox{insiemistiche};
sono alcuni esempi da cui la grammatica musicale si accinge ad un arricchimento ordinato di \textit{\mbox{oggetti sonori}}
%elenco note a pedice
\footnote{bibliografia di approfondimento: A. Schoemberg,\emph{Komposition mit 12 Tönen}; Pierre Boulez, \emph{Note di apprendistato} - Einaudi; J.Cage \emph{Lectures e writings}, Wesleian University Press; Iannis Xenakis, \emph{Musica e Architettura} - ed. Spirali; Ligeti, \emph{Metamorfosi della forma musicale} - EDT; G. Cantor, \emph{La formazione della teoria degli insiemi (scritti 1872-1899)} - Mimesis; Pierre Schaeffer, \emph{Trattato degli oggetti musicali}}
che definiscono una nuova tavolozza sintattica da cui descrivere varietà al dominio della forma.

Generalmente la comunicazione artistica si attua nel silenzio: la comprensione avviene per mezzo degli occhi e dell'orecchio senza bisogno di verbalizzazione.
A ciò, tuttavia, con approccio multidisciplinare le teorie linguistiche si intersecano alle analogie del linguaggio musicale con le nuove grammatiche trasformazionali\footnote{N. Chomsky \emph{La grammatica trasformazionale - Bollati Borghieri}}. Gli orizzonti della psicologia sperimentale cognitivista (come la psicologia della Gestald) a partire dalle riflessioni interdisciplianari di C. Stumpf\footnote{C. Stumpf, \emph{Die Anfänge der Musik - Le origini della musica}}, han elargito sempre più stimoli ad analisi e sperimentazioni pratiche sulle dinamiche che modellano tutte le percezioni\cite{rif3}. Per cogliere il senso tra i suoni, la coscienza musicale stabilisce dei legami, anticipando, completando, riempendo gli spazi del flusso temporale con l'immaginazione; a tale considerazione corrisponde anche e sopratutto una responsabilità sociale che ci prepari all'ascolto. La componente \virgo{culturale}, dunque, diviene fondamento essenziale per definire una semantica coerente ai fenomeni sonori\cite{rif1, rif3, rif5, rif7}. Le esperienze dell'ascoltore (del fuitore) possiedono un significato comunicabile sempre tenendo a conto che, la loro comunicabilità e il loro contenuto sono diversi a seconda del tipo di esperienza\cite{rif9}.
Si distinguono le emozioni estetiche dal contenuto emotivo: l'emozione estetica scaturisce da un \virgo{trionfo}, dal superamento di barriere quali la parola e l'ineffabilità di una realtà ma il contenuto emotivo può essere qualcosa di molto più profondo di ogni esperienza intellettuale, prerelazionale e vitale. Il piacere estetico, allora, è affine alla scoprire una \virgo{verità artistica} e di comprendere la forma simbolica presentazionale. I simboli artistici sono intraducibili singolarmente in quanto il loro \virgo{senso} è legato (collegato) alla particolare forma che hanno assunto.\cite{rif9}

La musica, essendo anche un evento fisico deve soddisfare anche una dimensione acustica\cite{rif4}. Si pone, a volte, un problema elementare di percezione: \virgo{in una grande sala, per esempio, il rumore d'ambiente, da solo, tende a coprire il livello dinamico della musica. La relazione psicologica con il pubblico, d'altra parte, si può stabilire, a una certa dimensione, soltanto disponendo un margine di tempo e di orecchio sufficiente; altrimenti, appena stabilito, il contatto si rompe e lo sforzo va rinnovato a ogni situazione per ricreare un \textit{circuito di audizione}}.\footnote{Pierre Boulez \textit{Note di apprendistato}}\\
Analizzando i processi cognitivi scaturiti dalla musica si cercano spiegazione attendibili a tutte le componenti del mestiere: dalla lettura o memorizzazione all'esecuzione, dalla composizione  all'ascolto.\cite{rif7}  
\subsection*{Una rappresentazione del silenzio musicale}
Il silenzio dell'artista accompagna il suo \virgo{fare}, il suo scolpire il suo scrivere. Il suo essere e la sua mente sono implicati nel creare, ma nel silenzio.
Nel corso scolastico, ho compreso che tutti i compositori, considerando perlomeno coloro che hanno contribuito a formare la \virgo{nostra} storia della musica hanno avuto a che fare con il silenzio ma, alcuni tra essi, hanno condiviso un'interpretazione di soggettiva profondità artistica, non solo ovvia condizione antitetica del suono bensì \textit{materia viva} ed immanente al suono stesso.\\
Theodor W. Adorno, musicologo, sociologo, pensatore e figura di spicco della \virgo{scuola di Francoforte}\footnote{nucleo, formato per lo più da filosofi e sociologi tedeschi nell'ambiente del \virgo{Istituto per la Ricerca Sociale} (Institut für Sozialforschung) della Uni Frankfurt;} aveva già abbozzato in un suo saggio del 1952-57\footnote{T.W.Adorno, \emph{Wagner} - Einaudi} un commento di risalto all'interpretazione espressiva del silenzio nel compositore Anton Webern: - \virgo{l'espressione, la soggettività viene assorbita e elevata a grado superiore nella costruzione} -. Tale osservazione analitica (che considerava W. con le regole del contrappunto e non per via matematico-statistica):  voleva dar luogo alla riflessione cui le pause non equivalgono più a \virgo{cesure}, a interpunzioni, ma diventano parte del tessuto sonoro: alle pause vengono attribuite qualità che alla musica tradizionale erano riservate al suono. I silenzi, in Webern, contribuiscono a generare \virgo{tensioni dinamiche} caratterizzanti. Evidenziare questo dettaglio qualitativo andava già a porsi in opposizione all'idea seriale per cui le pause erano semplici coefficienti negativi senza significato al di là della loro valore posizionale\cite{rif2}. Tale definizione venne rilevata ulteriormente da Franco Evangelisti che ne elevò a concetto di \mbox{\virgo{pre-apparizione} del silenzio}. Per Evangelisti, nella musica di Webern si infrange l 'idea di una continuità temporale del discorso musicale. I rapporti si capovolgono: prima la pausa significava \virgo{prendere respiro} in vista dell'esecuzione del suono, ora è il suono a \virgo{prendere respiro} nella pausa dove si condensano i contenuti esposti.\cite{rif3}\\
Interpretazione più intima è espressa da Artvo P\"art che assolve al silenzio funzioni precise quali dar maggior risalto alle articolazioni delle forme, sottolineare i contrasti timbrici e generare degli effetti espressivi di \virgo{attesa}\cite{rif6}.\\
\\
\begin{small}[...] io penetro quando cerco delle risposte - nella mia vita, nella mia musica, nel mio lavoro[...] ciò che è complesso e costituito da molteplici lati provoca confusione, e io devo cercare unità[...] tutto ciò che è senza importanza scompare.[...] Qui io sono solo con il silenzio. Ho scoperto che basta una nota sola, purché sia ben suonata. Questa nota, o la pausa silenziosa, o il momento di silenzio mi confortano. \textit{[Arvo Pärt in McCarty 1995]}
\end{small}\\

Per ragguagliare una così intima ideazione di \textit{silenzio sonoro} alla sua espressione artistica, P\"art, rivaluta i linguaggi delle correnti musicali del passato: in special modo dell'\textit{Ars Nova} del XVI secolo nella quale P. individua una musica senza contrasti, che gioca con le risonanze, fluida, perché libera dal tempo misurato; caratteristiche che anche la musica medioevale conosceva un profondo valore del silenzio\cite{rif6}.\\
Con Jhon Cage, \textit{l'enfant terrible} della musica degli anni sessanta-settanta\cite{rif6}, per la prima volta (1952) nella storia della musica occidentale è stato elaborato l'ossimoro di un \virgo{musicista silenzioso}. Nell'opera \textit{4'33''}, più che in ogni altra forma, il silenzio viene considerato materia sonora. Questo brano è considerato un simbolo, un messaggio in cui Cage dimostra ciò che è evidente: il silenzio autentico non esiste, e neppure più la musica\cite{rif8}. Durante quattro minuti e trentatré secondi i musicisti, senza emanare alcun suono dal proprio strumento eseguono l'indicazione \small{\textit{TACET}} della partitura: gestualità silenziose lasciano spazio ai suoni che passano nella sala o nel luogo d'ascolto.  E' una concatenazione in cui il fruitore (l'ascoltare) interagisce (seppur passivamente) con l'opera stessa. E' la fusione del silenzio con la realtà.\\
\begin{small}
 Grazie al silenzio, i rumori entrano definitivamente nella mia musica, e non una selezione di certi rumori, ma la molteplicità di tutti i rumori esistenti o che avvengono[...] \textit{[J. Cage, 1976]}
\end{small}

\subparagraph*{breve riflessione:}
Queste argomentazioni, affrontate ed approfondite nel percorso delle lezioni, promanano un fascino intenso, condizione che mi permette di elaborare con più consapevole slancio la visione critica cercando di non tralasciare nulla come scontato. L'importanza dell'approccio multidisciplinare, per quanto possa essere esoso in termini di ricerca analitica mi invita ad approfondire tematiche non direttamente subordinate alla materia musicale ma sempre ed in un certo qual modo tessute da un filo comune, la volontà di apprendere con metodo epistemologico qualsivoglia fenomeno. Ringrazio il docente di indirizzo, rigoroso e motivazionale che mi invita, in altra occasione, ad approfondire ulteriormente tali  argomentazioni trattate.
 
%BIBLIOGRAFIA ESSENZIALE
\begin{thebibliography}{}

\bibitem{rif1} 	1950, A. Copland, \emph{Come ascoltare la musica} - Garzanti;

\bibitem{rif2} 1985, dir. A. Basso, \emph{Dizionario della Musica} - UTET;

\bibitem{rif3} 1989, G. Borio M. Garda, \emph{L'Esperienza musicale: teoria e storia della ricezione} - EDT;

\bibitem{rif4} 1995, Tomatis, \emph{L'orecchio e il Linguaggio} - Ibis ;

\bibitem{rif5} 2007, di D. Schön, L. Akiva-Kabiri, Tomaso Vecchi, \emph{Psicologia della musica} - Carrocci editore;

\bibitem{rif6} 2001, D.Smoje,  \emph{L'udibile e l'inudibile \small{E. della musica}, Il novecento} - EINAUDI;

\bibitem{rif7} 1998, J.A.Sloboda, \emph{La mente musicale} \small{Saggi} - Il Mulino;

\bibitem{rif8} da '67 a '82, J.Cage, \emph{Lectures e writings}, Wesleian University Press;

\bibitem{rif9} 2005, D. Bertasio, M. Tessarolo, L. Verdi, M. R. Zorino, \emph{L'arte e il Silenzio}  \small{aspetti e problemi della comunicazione artistica - Guerini scientifica}.

\end{thebibliography}


\end{document}
